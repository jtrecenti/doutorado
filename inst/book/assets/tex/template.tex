% Arquivo LaTeX de exemplo de dissertação/tese a ser apresentada à CPG do IME-USP
%
% Criação: Jesús P. Mena-Chalco
% Revisão: Fabio Kon e Paulo Feofiloff
% Adaptação para UTF8, biblatex e outras melhorias: Nelson Lago
%
% Except where otherwise indicated, these files are distributed under
% the MIT Licence. The example text, which includes the tutorial and
% examples as well as the explanatory comments in the source, are
% available under the Creative Commons Attribution International
% Licence, v4.0 (CC-BY 4.0) - https://creativecommons.org/licenses/by/4.0/


%%%%%%%%%%%%%%%%%%%%%%%%%%%%%%%%%%%%%%%%%%%%%%%%%%%%%%%%%%%%%%%%%%%%%%%%%%%%%%%%
%%%%%%%%%%%%%%%%%%%%%%%%%%%%%%% PREÂMBULO LaTeX %%%%%%%%%%%%%%%%%%%%%%%%%%%%%%%%
%%%%%%%%%%%%%%%%%%%%%%%%%%%%%%%%%%%%%%%%%%%%%%%%%%%%%%%%%%%%%%%%%%%%%%%%%%%%%%%%

% A opção twoside (frente-e-verso) significa que a aparência das páginas pares
% e ímpares pode ser diferente. Por exemplo, as margens podem ser diferentes ou
% os números de página podem aparecer à direita ou à esquerda alternadamente.
% Mas nada impede que você crie um documento "só frente" e, ao imprimir, faça
% a impressão frente-e-verso.
%
% Aqui também definimos a língua padrão do documento
% (a última da lista) e línguas adicionais.
%\documentclass[12pt,twoside,brazilian,english]{book}
\documentclass[12pt,twoside,brazilian]{book}

% Ao invés de definir o tamanho das margens, vamos definir os tamanhos do
% texto, do cabeçalho e do rodapé, e deixamos a package geometry calcular
% o tamanho das margens em função do tamanho do papel. Assim, obtemos o
% mesmo resultado impresso, mas com margens diferentes, se o tamanho do
% papel for diferente.
\usepackage[a4paper]{geometry}

\geometry{
  textwidth=152mm,
  hmarginratio=12:17, % 24:34 -> com papel A4, 24mm + 152mm + 34mm = 210mm
  textheight=237mm,
  vmarginratio=8:7, % 32:28 -> com papel A4, 32mm + 237mm + 28mm = 297mm
  headsep=11mm, % distância entre a base do cabeçalho e o texto
  headheight=21mm, % qualquer medida grande o suficiente, p.ex., top - headsep
  footskip=10mm,
  marginpar=20mm,
  marginparsep=5mm,
}

% \usepackage{libertinus}
% \usepackage{libertinust1math}
% \usepackage{imagechapter}

% \usepackage[brazilian](babel)

% Vários pacotes e opções de configuração genéricos; para personalizar o
% resultado, modifique estes arquivos.
\input{assets/tex/extras/basics}
\input{assets/tex/extras/languages}
\input{assets/tex/extras/fonts}

\input{assets/tex/extras/floats}

\input{assets/tex/extras/imeusp-thesis} % capa, páginas preliminares e alguns detalhes
\input{assets/tex/extras/imeusp-formatting}
\input{assets/tex/extras/index}
\input{assets/tex/extras/bibconfig}
\input{assets/tex/extras/hyperlinks}
%\nocolorlinks % para impressão em P&B
\input{assets/tex/extras/source-code}
\input{assets/tex/extras/utils}

$if(highlighting-macros)$
$highlighting-macros$
$endif$


% Diretórios onde estão as figuras; com isso, não é preciso colocar o caminho
% completo em \includegraphics (e nem a extensão).
% \graphicspath{{figuras/},{logos/}}

% Comandos rápidos para mudar de língua:
% \en -> muda para o inglês
% \br -> muda para o português
% \texten{blah} -> o texto "blah" é em inglês
% \textbr{blah} -> o texto "blah" é em português
\babeltags{br = brazilian, en = english}

% Bibliografia
\usepackage[
  style=assets/tex/extras/plainnat-ime, % variante de autor-data, similar a plainnat
  %style=alphabetic, % similar a alpha
  %style=numeric, % comum em artigos
  %style=authoryear-comp, % autor-data "padrão" do biblatex
  %style=apa, % variante de autor-data, muito usado
  %style=abnt,
]{biblatex}

\usepackage{bookmark}
\usepackage{hhline}

\providecommand{\tightlist}{%
  \setlength{\itemsep}{0pt}\setlength{\parskip}{0pt}}

%%%%%%%%%%%%%%%%%%%%%%%%%%%%%%%%%%%%%%%%%%%%%%%%%%%%%%%%%%%%%%%%%%%%%%%%%%%%%%%%
%%%%%%%%%%%%%%%%%%%%%%%%%%%%%%%%%% METADADOS %%%%%%%%%%%%%%%%%%%%%%%%%%%%%%%%%%%
%%%%%%%%%%%%%%%%%%%%%%%%%%%%%%%%%%%%%%%%%%%%%%%%%%%%%%%%%%%%%%%%%%%%%%%%%%%%%%%%

% O arquivo com os dados bibliográficos para biblatex; você pode usar
% este comando mais de uma vez para acrescentar múltiplos arquivos
\addbibresource{assets/bib/book.bib}

% Este comando permite acrescentar itens à lista de referências sem incluir
% uma referência de fato no texto (pode ser usado em qualquer lugar do texto)
%\nocite{bronevetsky02,schmidt03:MSc, FSF:GNU-GPL, CORBA:spec, MenaChalco08}
% Com este comando, todos os itens do arquivo .bib são incluídos na lista
% de referências
%\nocite{*}

% É possível definir como determinadas palavras podem (ou não) ser
% hifenizadas; no entanto, a hifenização automática geralmente funciona bem
% \babelhyphenation{documentclass latexmk soft-ware clsguide} % todas as línguas
% \babelhyphenation[brazilian]{Fu-la-no}
% \babelhyphenation[english]{what-ever}

% Estes comandos definem o título e autoria do trabalho e devem sempre ser
% definidos, pois além de serem utilizados para criar a capa, também são
% armazenados nos metadados do PDF.
\title{
    % Obrigatório nas duas línguas
    titlept={Resolvendo Captchas},
    titleen={Solving Captchas},
    % Opcional, mas se houver deve existir nas duas línguas
    subtitlept={usando aprendizado parcialmente supervisionado},
    subtitleen={using partial supervised learning},
}

\author{Julio Adolfo Zucon Trecenti}

% Para TCCs, este comando define o supervisor
\orientador{Prof. Dr. Victor Fossaluza}

% Se não houver, remova; se houver mais de um, basta
% repetir o comando quantas vezes forem necessárias
% \coorientador{Prof. Dr. Ciclano de Tal}
% \coorientador[fem]{Profª. Drª. Beltrana de Tal}

% A página de rosto da versão para depósito (ou seja, a versão final
% antes da defesa) deve ser diferente da página de rosto da versão
% definitiva (ou seja, a versão final após a incorporação das sugestões
% da banca).
\defesa{
  nivel=doutorado, % mestrado, doutorado ou tcc
  % É a versão para defesa ou a versão definitiva?
  %definitiva,
  % É qualificação?
  %quali,
  programa={Estatística},
  membrobanca={Prof. Dr. Victor Fossaluza (orientadora) -- IME-USP [sem ponto final]},
  % Em inglês, não há o "ª"
  %membrobanca{Prof. Dr. Fulana de Tal (advisor) -- IME-USP [sem ponto final]},
  membrobanca={Prof. Dr. Rafael Izbicki -- UFSCar [sem ponto final]},
  membrobanca={Prof. Dr. Rafael Stern -- IME-USP [sem ponto final]},
  membrobanca={Prof. Dr. Renato Vicente -- IME-USP [sem ponto final]},
  % Se não houve bolsa, remova
  %
  % Norma sobre agradecimento por auxílios da FAPESP:
  % https://fapesp.br/11789/referencia-ao-apoio-da-fapesp-em-todas-as-formas-de-divulgacao
  %
  % Norma sobre agradecimento por auxílios da CAPES (Portaria 206,
  % de 4 de Setembro de 2018):
  % https://www.in.gov.br/materia/-/asset_publisher/Kujrw0TZC2Mb/content/id/39729251/do1-2018-09-05-portaria-n-206-de-4-de-setembro-de-2018-39729135
  %
  %apoio={O presente trabalho foi realizado com apoio da Coordenação
  %       de Aperfeiçoamento\\ de Pessoal de Nível Superior -- Brasil
  %       (CAPES) -- Código de Financiamento 001}, % o código é sempre 001
  %
  %apoio={This study was financed in part by the Coordenação de
  %       Aperfeiçoamento\\ de Pessoal de Nível Superior -- Brasil
  %       (CAPES) -- Finance Code 001}, % o código é sempre 001
  %
  %apoio={Durante o desenvolvimento deste trabalho, o autor recebeu\\
  %       auxílio financeiro da FAPESP -- processo nº aaaa/nnnnn-d},
  %
  %apoio={During the development if this work, the author received\\
  %       financial support from FAPESP -- grant \#aaaa/nnnnn-d},
  %
  % apoio={Durante o desenvolvimento deste trabalho o autor
  %        recebeu auxílio financeiro da XXXX},
  % local={São Paulo},
  % data=2017-08-10, % YYYY-MM-DD
  % A licença do seu trabalho. Use CC-BY, CC-BY-NC, CC-BY-ND, CC-BY-SA,
  % CC-BY-NC-SA ou CC-BY-NC-ND para escolher a licença Creative Commons
  % correspondente (o sistema insere automaticamente o texto da licença).
  % Se quiser estabelecer regras diferentes para o uso de seu trabalho,
  % converse com seu orientador e coloque o texto da licença aqui, mas
  % observe que apenas TCCs sob alguma licença Creative Commons serão
  % acrescentados ao BDTA. Se você tem alguma intenção de publicar o
  % trabalho comercialmente no futuro, sugerimos a licença CC-BY-NC-ND.
  direitos={CC-BY}, % Creative Commons Attribution 4.0 International License
  %direitos={CC-BY-NC-ND}, % Creative Commons Attribution / NonCommercial /
                           % NoDerivatives 4.0 International License
  %direitos={Autorizo a reprodução e divulgação total ou parcial
  %          deste trabalho, por qualquer meio convencional ou
  %          eletrônico, para fins de estudo e pesquisa, desde que
  %          citada a fonte.},
  %direitos={I authorize the complete or partial reproduction and disclosure
  %          of this work by any conventional or electronic means for study
  %          and research purposes, provided that the source is acknowledged.}
  % Para gerar a ficha catalográfica, acesse https://fc.ime.usp.br/,
  % preencha o formulário e escolha a opção "Gerar Código LaTeX".
  % Basta copiar e colar o resultado aqui.
  fichacatalografica={},
}


\newlength{\cslhangindent}
\setlength{\cslhangindent}{1.5em}
\newlength{\csllabelwidth}
\setlength{\csllabelwidth}{3em}
\newlength{\cslentryspacingunit} % times entry-spacing
\setlength{\cslentryspacingunit}{\parskip}
\newenvironment{CSLReferences}[2] % #1 hanging-ident, #2 entry spacing
 {% don't indent paragraphs
  \setlength{\parindent}{0pt}
  % turn on hanging indent if param 1 is 1
  \ifodd #1
  \let\oldpar\par
  \def\par{\hangindent=\cslhangindent\oldpar}
  \fi
  % set entry spacing
  \setlength{\parskip}{#2\cslentryspacingunit}
 }%
 {}
\usepackage{calc}
\newcommand{\CSLBlock}[1]{#1\hfill\break}
\newcommand{\CSLLeftMargin}[1]{\parbox[t]{\csllabelwidth}{#1}}
\newcommand{\CSLRightInline}[1]{\parbox[t]{\linewidth - \csllabelwidth}{#1}\break}
\newcommand{\CSLIndent}[1]{\hspace{\cslhangindent}#1}


\setkeys{Gin}{width=0.8\textwidth,height=0.8\textheight,keepaspectratio}

%%%%%%%%%%%%%%%%%%%%%%%%%%%%%%%%%%%%%%%%%%%%%%%%%%%%%%%%%%%%%%%%%%%%%%%%%%%%%%%%
%%%%%%%%%%%%%%%%%%%%%%% AQUI COMEÇA O CONTEÚDO DE FATO %%%%%%%%%%%%%%%%%%%%%%%%%
%%%%%%%%%%%%%%%%%%%%%%%%%%%%%%%%%%%%%%%%%%%%%%%%%%%%%%%%%%%%%%%%%%%%%%%%%%%%%%%%


\begin{document}

%%%%%%%%%%%%%%%%%%%%%%%%%%% CAPA E PÁGINAS INICIAIS %%%%%%%%%%%%%%%%%%%%%%%%%%%%

% Aqui começa o conteúdo inicial que aparece antes do capítulo 1, ou seja,
% página de rosto, resumo, sumário etc. O comando frontmatter faz números
% de página aparecem em algarismos romanos ao invés de arábicos e
% desabilita a contagem de capítulos.
\frontmatter

\pagestyle{plain}

\onehalfspacing % Espaçamento 1,5 na capa e páginas iniciais

\maketitle % capa e folha de rosto

%%%%%%%%%%%%%%%% DEDICATÓRIA, AGRADECIMENTOS, RESUMO/ABSTRACT %%%%%%%%%%%%%%%%%%

\begin{dedicatoria}
Aos meus pais,
\end{dedicatoria}

% Reinicia o contador de páginas (a próxima página recebe o número "i") para
% que a página da dedicatória não seja contada.
\pagenumbering{roman}

% Agradecimentos:
% Se o candidato não quer fazer agradecimentos, deve simplesmente eliminar
% esta página. A epígrafe, obviamente, é opcional; é possível colocar
% epígrafes em todos os capítulos. O comando "\chapter*" faz esta seção
% não ser incluída no sumário.
\chapter*{Agradecimentos}

Escrever a parte de agradecimento é uma tarefa muito fácil, pois o sentimento de gratidão que sinto a todas as pessoas queridas estiveram ao meu lado no processo de construção da tese é muito forte e verdadeiro. Ao mesmo tempo, é uma tarefa muito difícil, pois acredito não sou capaz de expressar em palavras o tamanho dessa gratidão e o quanto o apoio dessas pessoas significaram para mim.

Primeiramente, gostaria de agradecer ao Victor Fossaluza, meu querido amigo e orientador. Victor foi um irmão mais velho, com a paciência necessária para lidar com minhas limitações por fazer um doutorado sem bolsa e trabalhando muito. Foi amigo, fazendo o trabalho motivacional para me convencer de que eu estava fazendo algo relevante. E foi um grande sábio, dando sugestões muito interessantes para resolver os desafios que foram postos.

Agradeço ao Daniel Falbel, não só pela amizade, mas por ter me salvado este trabalho pelo menos duas vezes. Se não fosse pelo trabalho do Daniel, o trabalho nem teria começado, já que foi ele que deu a ideia de modelar Captchas usando deep learning pela primeira vez, usando o TensorFlow. E se não fosse pelo maravilhoso trabalho em na família de pacotes em torno do torch, como o próprio torch, luz e torchvision.

Agradeço ao Caio Lente, por ser essa pessoa incrível, gentil e modesta, apesar da enorme capacidade. Caio é a única pessoa capaz de me aturar como sócio em duas empresas diferentes, a Curso-R e a Terranova. Caio foi a pessoa que desenvolveu o pacote e o site do decryptr, nossa primeira ferramenta geral de resolver Captchas. Caio também é meu influencer digital, sendo a pessoa por trás de coisas que acompanham minha vida até hoje, como o Todoist e o podcast Philosophize This.

Agradeço à minha noiva Beatriz Milz, por todo o apoio, dedicação e paciência que teve comigo, especialmente nas etapas finais do desenvolvimento da tese. Beatriz é a melhor pessoa que conheço e tenho muita sorte de poder viver ao seu lado. Agradeço pelos excelentes comentários que colocou na tese (é possível \href{https://github.com/jtrecenti/doutorado/pulls?q=is\%3Apr}{visualizá-los no repositório da tese no GitHub}) e pela ajuda na tomada de várias decisões ao longo da construção da tese.

Agradeço ao Fernando Corrêa, por tantas coisas que é difícil de expressar. Fernando foi quem me convenceu a desenvolver o doutorado em torno dos Captchas, em uma conversa que tivemos no carro. Cada conversa com Fernando é um aprendizado novo, seja sobre as filosofias mais profundas, os problemas mais importantes da humanidade ou sobre os personagens mais engraçados do BBB.

Agradeço ao Athos Damiani, meu grande amigo, colega de TCC e \textit{roomate}, pelas grandes discussões. O Athos é uma pessoa maravilhosa de discutir, pois ele não desiste! Isso fez com que eu precisasse entender mais dos assuntos para argumentar, percebendo que na verdade eu não estava entendendo tanto. Além disso, Athos foi quem me ensinou a resolver Captchas de audio, o que me ajudou muito na construção das soluções.

Para completar o grupo de sócios da Curso-R, agradeço imensamente ao William Amorim, meu grande amigo e parceiro no processo de doutoramento. Talvez ele não saiba, mas sempre o pilar que me apoiou nas horas difíceis, já que terminou o doutorado antes de mim e me mostrou empiricamente que isso era possível.

Agradeço também a todas as pessoas das outras empresas que faço parte: ABJ, Insper e Terranova. O suporte de vocês foi fundamental para que eu conseguisse tempo para desenvolver a tese. Agradecimentos especiais à Barbara Tassoni, Igor Pretel, Marcelo Guedes Nunes, Ricardo Feliz e Renata Hirota.

Também gostaria de agradecer a uma instituição, o Instituto de Matemática e Estatística da Universidade de São Paulo. Fazer a graduação, mestrado e doutorado nessa instituição foi uma experiência transformadora. Espero poder retribuir pelo menos em parte tudo o que o maravilhoso universo da estatística me proporcionou. Agradeço a todos os professores, alunos e funcionários que fizeram e continuam fazendo parte dessa história.

Finalmente, agradeço aos meus pais, Cidimir e Vera, e ao meu irmão Lucas, por estarem sempre me apoiando e me cobrando sobre o doutorado. Sem esse apoio, provavelmente teria desistido no meio o caminho.

% \input{conteudo/resumoabstract}


%%%%%%%%%%%%%%%%%%%%%%%%%%% LISTAS DE FIGURAS ETC. %%%%%%%%%%%%%%%%%%%%%%%%%%%%%

% Como as listas que se seguem podem não incluir uma quebra de página
% obrigatória, inserimos uma quebra manualmente aqui.
\makeatletter
\if@openright\cleardoublepage\else\clearpage\fi
\makeatother

% Todas as listas são opcionais; Usando "\chapter*" elas não são incluídas
% no sumário. As listas geradas automaticamente também não são incluídas por
% conta das opções "notlot" e "notlof" que usamos para a package tocbibind.

% Normalmente, "\chapter*" faz o novo capítulo iniciar em uma nova página, e as
% listas geradas automaticamente também por padrão ficam em páginas separadas.
% Como cada uma destas listas é muito curta, não faz muito sentido fazer isso
% aqui, então usamos este comando para desabilitar essas quebras de página.
% Se você preferir, comente as linhas com esse comando e des-comente as linhas
% sem ele para criar as listas em páginas separadas. Observe que você também
% pode inserir quebras de página manualmente (com \clearpage, veja o exemplo
% mais abaixo).
\newcommand\disablenewpage[1]{{\let\clearpage\par\let\cleardoublepage\par #1}}

% Nestas listas, é melhor usar "raggedbottom" (veja basics.tex). Colocamos
% a opção correspondente e as listas dentro de um grupo para ativar
% raggedbottom apenas temporariamente.
\bgroup
\raggedbottom

%%%%% Listas criadas manualmente

%\chapter*{Lista de abreviaturas}
\disablenewpage{\chapter*{Lista de siglas}}

\begin{tabular}{rl}
   ABJ & Associação Brasileira de Jurimetria\\
   ADAM & ADaptive Moment Estimator\\
   API & Application Programming Interface\\
   CADESP & Centro de Apoio ao Desenvolvimento da Saúde Pública\\
   Captcha & Completely Automated Public Turing test to tell Computers and Humans Apart\\
   CF & Constituição Federal\\
   CLT & Consolidação das Leis do Trabalho\\
   CKAN & Comprehensive Knowledge Archive Network\\
   CNJ & Conselho Nacional de Justiça\\
   CNN & Convolutional Neural Networks\\
   CNPJ & Cadastro Nacional da Pessoa Jurídica\\
   CSV & Comma Separated Values\\
   ENCE & Escola Nacional de Ciências Estatísticas\\
   HIP & Human Interaction Proofs\\
   HTTP & HypertText Transfer Protocol\\
   HTML & HyperText Markup Language\\
   IETF & Internet Engineering Task Force\\
   JPEG & Join Photographic Experts Groups\\
   GAN & Generative Adversarial Networks\\
   GPT & Generative Pre-Training Transformer\\
   JUCESP & Junta Comercial de São Paulo\\
   LAI & Lei de Acesso à Informação\\
   ICMC & Instituto de Ciências Matemáticas e de Computação\\
   IME & Instituto de Matemática e Estatística\\
   LGPD & Lei Geral de Proteção de Dados\\
   ME & Ministério da Economia\\
\end{tabular}

\begin{tabular}{rl}
   MNIST & Modified National Institute of Standards and Technology Database\\
   OCR & Optical Character Recognition\\
   OKFN & Open Knowledge Foundation\\
   PDF & Portable Document Format\\
   PLL & Partial Label Learning\\
   PNG & Portable Network Graphics\\
   RCN & Recursive Cortical Network\\
   RFB & Receita Federal do Brasil\\
   ReLU & Rectified Linear Unit\\
   SAJ & Sistema de Automação da Justiça\\
   SEI & Sistema Eletrônico de Informações\\
   SGT & Sistema de Gestão de Tabelas\\
   SPAM & Sending and Posting Advertisement in Mass\\
   TJBA & Tribunal de Justiça da Bahia\\
   TJMG & Tribunal de Justiça de Minas Gerais\\
   TJPE & Tribunal de Justiça de Pernambuco\\
   TJRS & Tribunal de Justiça do Rio Grande do Sul\\
   TJSP & Tribunal de Justiça de São Paulo\\
   TRF & Tribunal Regional Federal\\
   TRT & Tribunal Regional do Trabalho\\
   UFAM & Universidade Federal do Amazonas\\
   UFBA & Universidade Federal da Bahia\\
   UFF & Universidade Federal Fluminense\\
   UFG & Universidade Federal de Goiás\\
   UFMG & Universidade Federal de Minas Gerais\\
   UFPE & Universidade Federal de Pernambuco\\
   UFPR & Universidade Federal do Paraná\\
   UFRGS & Universidade Federal do Rio Grande do Sul\\
   UFRJ & Universidade Federal do Rio de Janeiro\\
   UFRN & Universidade Federal do Rio Grande do Norte\\
   UFSCar & Universidade Federal de São Carlos\\
   UnB & Universidade de Brasília\\
   Unicamp & Universidade Estadual de Campinas\\
   UNESP & Universidade Estadual de São Paulo\\
   USP & Universidade de São Paulo\\
   WAWL & Web Automatic Weak Learner\\
   XPath & XML Path Language\\
   XML & eXtensible Markup Language
\end{tabular}

\clearpage

%\chapter*{Lista de símbolos}
\disablenewpage{\chapter*{Lista de símbolos}}

% Quebra de página manual
\clearpage

%%%%% Listas criadas automaticamente

% Você pode escolher se quer ou não permitir a quebra de página
%\listoffigures
\disablenewpage{\listoffigures}

% Você pode escolher se quer ou não permitir a quebra de página
%\listoftables
\disablenewpage{\listoftables}

% Esta lista é criada "automaticamente" pela package float quando
% definimos o novo tipo de float "program" (em utils.tex)
% Você pode escolher se quer ou não permitir a quebra de página
%\listof{program}{\programlistname}
% \disablenewpage{\listof{program}{\programlistname}}

% Sumário (obrigatório)
\tableofcontents

\egroup % Final de "raggedbottom"

% Referências indiretas ("x", veja "y") para o índice remissivo (opcionais,
% pois o índice é opcional). É comum colocar esses itens no final do documento,
% junto com o comando \printindex, mas em alguns casos isso torna necessário
% executar texindy (ou makeindex) mais de uma vez, então colocar aqui é melhor.
% \index{Inglês|see{Língua estrangeira}}
% \index{Figuras|see{Floats}}
% \index{Tabelas|see{Floats}}
% \index{Código-fonte|see{Floats}}
% \index{Subcaptions|see{Subfiguras}}
% \index{Sublegendas|see{Subfiguras}}
% \index{Equações|see{Modo matemático}}
% \index{Fórmulas|see{Modo matemático}}
% \index{Rodapé, notas|see{Notas de rodapé}}
% \index{Captions|see{Legendas}}
% \index{Versão original|see{Tese/Dissertação, versões}}
% \index{Versão corrigida|see{Tese/Dissertação, versões}}
% \index{Palavras estrangeiras|see{Língua estrangeira}}
% \index{Floats!Algoritmo|see{Floats, ordem}}


%%%%%%%%%%%%%%%%%%%%%%%%%%%%%%%% CAPÍTULOS %%%%%%%%%%%%%%%%%%%%%%%%%%%%%%%%%%%%%

% Aqui vai o conteúdo principal do trabalho, ou seja, os capítulos que compõem
% a dissertação/tese. O comando mainmatter reinicia a contagem de páginas,
% modifica a numeração para números arábicos e ativa a contagem de capítulos.
\mainmatter

\pagestyle{mainmatter}

% Espaçamento simples
\singlespacing

$body$

% \input{conteudo/00-exemplo-introducao}
% \input{conteudo/01-exemplo-normas-ime}
% \input{conteudo/02-exemplo-usando-o-modelo}
% \input{conteudo/03-exemplo-latex}
% \input{conteudo/04-exemplo-tutorial}
% \input{conteudo/05-exemplo-exemplos}


%%%%%%%%%%%%%%%%%%%%%%%%%%%% APÊNDICES E ANEXOS %%%%%%%%%%%%%%%%%%%%%%%%%%%%%%%%

% Um apêndice é algum conteúdo adicional de sua autoria que faz parte e
% colabora com a ideia geral do texto mas que, por alguma razão, não precisa
% fazer parte da sequência do discurso; por exemplo, a demonstração de um
% teorema intermediário, as perguntas usadas em uma pesquisa qualitativa etc.
%
% Um anexo é um documento que não faz parte da tese (em geral, nem é de sua
% autoria) mas é relevante para o conteúdo; por exemplo, a especificação do
% padrão técnico ou a legislação que o trabalho discute, um artigo de jornal
% apresentando a percepção do público sobre o tema da tese etc.
%
% Os comandos appendix e annex reiniciam a numeração de capítulos e passam
% a numerá-los com letras. "annex" não faz parte de nenhuma classe padrão,
% foi criado para este modelo. Se o trabalho não tiver apêndices ou anexos,
% remova estas linhas.
%
% Diferentemente de \mainmatter, \backmatter etc., \appendix e \annex não
% forçam o início de uma nova página. Em geral isso não é importante, pois
% o comando seguinte costuma ser "\chapter", mas pode causar problemas com
% a formatação dos cabeçalhos. Assim, vamos forçar uma nova página antes
% de cada um deles.

%%%% Apêndices %%%%

\makeatletter
\if@openright\cleardoublepage\else\clearpage\fi
\makeatother

% \pagestyle{appendix}

% \appendix

% \addappheadtotoc acrescenta a palavra "Apêndice" ao sumário; se
% só há apêndices, sem anexos, provavelmente não é necessário.
% \addappheadtotoc

% \input{conteudo/apendice-exemplo-pseudocodigo}
% \par

%%%% Anexos %%%%

% \makeatletter
% \if@openright\cleardoublepage\else\clearpage\fi
% \makeatother

% \pagestyle{appendix} % repete o anterior, caso você não use apêndices

% \annex

% \addappheadtotoc acrescenta a palavra "Anexo" ao sumário; se
% só há anexos, sem apêndices, provavelmente não é necessário.
% \addappheadtotoc

% \input{conteudo/anexo-exemplo-faq}
% \par


%%%%%%%%%%%%%%% SEÇÕES FINAIS (BIBLIOGRAFIA E ÍNDICE REMISSIVO) %%%%%%%%%%%%%%%%

% O comando backmatter desabilita a numeração de capítulos.
\backmatter

\pagestyle{backmatter}

% Espaço adicional no sumário antes das referências / índice remissivo
\addtocontents{toc}{\vspace{2\baselineskip plus .5\baselineskip minus .5\baselineskip}}

% A bibliografia é obrigatória

\printbibliography[
  title=\refname\label{bibliografia}, % "Referências", recomendado pela ABNT
  %title=\bibname\label{bibliografia}, % "Bibliografia"
  heading=bibintoc, % Inclui a bibliografia no sumário
]

% \printindex % imprime o índice remissivo no documento (opcional)

\end{document}
